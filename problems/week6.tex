\documentclass[11pt]{amsart}


\usepackage{etoolbox}
\newtoggle{answers}
\togglefalse{answers}

% Comment out the following line to not show answers.
\toggletrue{answers}




\usepackage{amssymb,amsmath,amsfonts,amsthm,mathrsfs,fullpage}
%,amsmath,amsfonts,amsthm,mathrsfs}
\usepackage[all]{xy}  % for commutative diagrams
\SelectTips{cm}{10}
\usepackage{mathrsfs,float}
\numberwithin{equation}{section}

\usepackage{tikz}
\usetikzlibrary{arrows,shapes,positioning,calc,patterns,cd}


% Color comments
\usepackage{color}

%\definecolor{backgroundcolor}{rgb}{1,1,0.8}
%\pagecolor{backgroundcolor}

\newcommand{\commentr}[1]{{\color{red} [#1]}}
\newcommand{\commentb}[1]{{\color{blue} [#1]}}
\newcommand{\commentm}[1]{{\color{magenta} [#1]}}

\newcommand{\bnum}{\begin{enumerate}} %Use with \item to create numbered lists
\newcommand{\enum}{\end{enumerate}}
\newcommand{\babc}{\bnum\renewcommand{\labelenumi}{(\alph{enumi})}}%Use with \item for abc lists
\newcommand{\eabc}{\end{enumerate}}
\newcommand{\bijk}{\bnum\renewcommand{\labelenmi}{\roman{enumi}.}}%Roman numeral lists
\newcommand{\eijk}{\end{enumerate}}

\usepackage{verbatim,moreverb}

\newcommand{\kk}{\mathbb k}
\renewcommand{\AA}{\mathbb A}
\newcommand{\BB}{\mathbb B}
\newcommand{\CC}{\mathbb C}
\newcommand{\DD}{\mathbb D}
\newcommand{\EE}{\mathbb E}
\newcommand{\FF}{\mathbb F}
\newcommand{\GG}{\mathbb G}
\newcommand{\HH}{\mathbb H}
\newcommand{\II}{\mathbb I}
\newcommand{\JJ}{\mathbb J}
\newcommand{\KK}{\mathbb K}
\newcommand{\LL}{\mathbb L}
\newcommand{\MM}{\mathbb M}
\newcommand{\NN}{\mathbb N}
\newcommand{\PP}{\mathbb P}
\newcommand{\QQ}{\mathbb Q}
\newcommand{\RR}{\mathbb R}
\renewcommand{\SS}{\mathbb S}
\newcommand{\TT}{\mathbb T}
\newcommand{\UU}{\mathbb U}
\newcommand{\VV}{\mathbb V}
\newcommand{\WW}{\mathbb W}
\newcommand{\XX}{\mathbb X}
\newcommand{\YY}{\mathbb Y}
\newcommand{\ZZ}{\mathbb Z} 

\newcommand{\Zhat}{\widehat\ZZ}

\newcommand{\calA}{\mathcal A} 
\newcommand{\C}{\mathcal C} 

\newcommand{\OO}{\mathcal O}
\newcommand{\F}{\mathscr F} %\renewcommand{\P}{\mathscr P}
\newcommand{\Hh}{\mathscr H} 
\newcommand{\N}{\mathscr N}\newcommand{\Ii}{\mathscr I}
\newcommand{\Z}{\mathscr Z}
\newcommand{\bb}{\mathfrak b}\newcommand{\m}{\mathfrak m}\newcommand{\M}{\mathcal M}
\newcommand{\aA}{\mathfrak a} \newcommand{\fF}{\mathfrak f}
\newcommand{\qq}{\mathfrak q} 
\newcommand{\p}{\mathfrak p} \newcommand{\Pp}{\mathfrak P}
\newcommand{\norm}[1]{ \left|\left| #1 \right|\right|  }
\newcommand{\ang}[1]{ \langle #1 \rangle  }
\newcommand{\aside}[1]{ \marginpar{#1} }
\newcommand\legendre[2]{\Bigl(\frac{#1}{#2}\Bigr) }   \newcommand\Angle[2]{\langle #1,#2 \rangle}


\def\Spec{\operatorname{Spec}} \def\id{\operatorname{id}}
\def\Div{\operatorname{Div}}\def\tr{\operatorname{tr}}
\def\Supp{\operatorname{Supp}} \def\Gal{\operatorname{Gal}}

\def \GL {\operatorname{GL}_2}  
\def \PGL {\operatorname{PGL}_2}
\def \SL {\operatorname{SL}_2}
\def \PSL {\operatorname{PSL}_2}

\def\Res{\operatorname{Res}}
\def\Aut{\operatorname{Aut}} \def\End{\operatorname{End}}

\def\Prim{\operatorname{Prim}} \def\Fr{\operatorname{Frob}}
\def\lcm{\operatorname{lcm}} \def\Li{\operatorname{Li}}
\newcommand{\Hom}{\operatorname{Hom}}
\newcommand{\Card}{\operatorname{Card}}
\newcommand{\coker}{\operatorname{coker}}
\newcommand{\coimage}{\operatorname{coimage}}
\newcommand{\image}{\operatorname{image}}
\newcommand{\Ext}{\operatorname{Ext}}
\def \Ev {\operatorname{Ev}}
\def \rad {\operatorname{rad}}

\def \G {\mathcal G}
\def \B {\mathcal B}





\def\power{\mathcal{P}}
\def\Xx{\mathscr X} \def\Zz{\mathscr Z} \def\Ss{\mathscr S} \def\Gg{\mathscr G}
\def\Rr{\mathscr R}\def\Dd{\mathcal D}
 \def\cC{\mathfrak c}
\def\join{\vee}
\def\meet{\wedge}
\def\ord{\operatorname{ord}}

\def\tors{\operatorname{tors}}
\def\sfree{\operatorname{sf}}


\def\bbar#1{\setbox0=\hbox{$#1$}\dimen0=.2\ht0 \kern\dimen0 \overline{\kern-\dimen0 #1}}
\newcommand{\Qbar}{{\overline{\mathbb Q}}} 
\newcommand{\Kbar}{\bbar{K}} 
\newcommand{\kbar}{\bbar{k}} 
\newcommand{\Fbar}{\bbar{F}} 
\newcommand{\FFbar}{\overline{\FF}} 

\newcommand{\smallpmod}[1]{\text{ }(\operatorname{mod } #1 )}

\newcommand{\defi}[1]{\textsf{#1}} % for defined terms

\newtheorem{thm}{Theorem}[section]
\newtheorem{lemma}[thm]{Lemma}
\newtheorem{cor}[thm]{Corollary}
\newtheorem{prop}[thm]{Proposition}
\newtheorem{conj}[thm]{Conjecture} 
\newtheorem{example}[thm]{Example}
\newtheorem{claim}{Claim}

\theoremstyle{definition}
\newtheorem{definition}[thm]{Definition}

\theoremstyle{remark}
\newtheorem{remark}[thm]{Remark}
\newtheorem{remarks}[thm]{Remarks}


\newenvironment{romanenum}{\hfill \begin{enumerate} \renewcommand{\theenumi}{\roman{enumi}}}{\end{enumerate}}
\renewcommand{\Re}{\operatorname{Re}}

\definecolor{webbrown}{rgb}{.6,0,0}
\usepackage[
   %    draft,
        colorlinks,
        linkcolor=webbrown,  filecolor=webcolor,  citecolor=webbrown, 
        backref,
        %pdfauthor={David Zywina}, % add other authors
    %   pdftitle={Paper title goes here},
]{hyperref}
\usepackage[alphabetic,backrefs,lite]{amsrefs} % for bibliography
%Uncomment to double space:
%\renewcommand{\baselinestretch}{2}

\newcounter{excount}
\newcommand\exercise[1]{\addtocounter{excount}{1}\noindent\textbf{Problem \arabic{excount}: }#1{$ $}\\}

\newcommand\answer[1]{
\iftoggle{answers}{{\noindent \color{blue} #1\\ }}{ } 
}


\begin{document}
\setcounter{excount}{0}

\title{CoCAAG, Winter 2025: Week 6 Questions}
\maketitle
\vspace{0.5cm}
\noindent \textbf{During ``project time'',
  work through a jupyter notebook, try exercises, or do these!}

\medskip

In some of the problems, you are asked to try your methods on one or some of the following varieties:
\babc
\item the cubic surface $X$ in $\PP^3$: $x^3+y^3+z^3+w^3 = 0$
\item the Fermat quartic surface   $x^4+y^4+z^4+w^4=0$, this is what is called a K3 surface,
\item the variety $X \subset \PP^4$ whose ideal is given by the 3 by 3 minors of a
  (fairly random) $4 \times 3$ matrix of linear forms in 5 variables.
\item Other possibly random hypersurfaces in $\PP^n$ of degree $d$, e.g. $(n,d) = (4,5), (4,4), (4,6)$.
  \eabc
  
\bigskip

\exercise{
  Consider the special case $\PP^2$.  Compute from the definition in class the cohomologies of $\OO_{\PP^2}(d)$, for all $d$,
  verifying Serre's theorem in this case.  Hint: this complex with infinitely generated modules is (over the base field)
  the direct sum of complexes corresponding to each monomial $x_0^a x_1^b x_2^c$, for $(a,b,c) \in \ZZ^3$.
  Once one sees the proof in this special case, it is pretty easy to generalize to prove the entire theorem (and, in fact,
  even more, that we haven't stated yet!).
}

\exercise{
  In this exercise we prove the important theorems of Serre ((b), (c)).
  After proving (a),
  use Serre's ``local duality'' theorem to prove (b), (c).
  Here, suppose that $\widetilde{M}$ is
  a coherent sheaf on $\PP^n$ where $M$ is a finitely generated graded
  $S$-module.
  \babc
\item Show that the $k$-dual of $M$ is zero in all degrees $d \gg 0$.
\item Show that $H^i(\OO_X(d)) = 0$, for $i > 0$ and $d \gg 0$.
\item Show that $\OO_X(d)$ is generated by global sections for $d \gg 0$.  This means that the
  natural map $M_d \to H^0(\OO_X(d))$ is an isomorphism.
  \eabc
}

\exercise{
  Consider the projective variety $X$ which is given by the zeros of $x^3 + y^3 + z^3$, in $\PP^2$ (this is an elliptic curve).
  \babc
\item Find the dimensions of the cohomology vector spaces of $\OO_X$.
  \item Compute ``by hand'' a module which sheafifies to $\Omega^1_X$.
    \item Find the dimensions of the cohomology vector spaces of 1-forms $\Omega^1_X$.
      \eabc
}

\exercise{
  Compute the sheaf cohomology of the sheaf associated to $\operatorname{Hom}(I/I^2, S/I)$, for some choices of monomial ideals $I$ in a polynomial ring $S$.
  (start with e.g. $I = (x^2, xy, y^2) \subset k[x,y,z]$).
  The $H^0$ of this turns out to be the tangent space to the point $[I]$ on its Hilbert scheme.
  }

\exercise{
  Consider the projective variety $X$ which is given by the 3 by 3 minors of a (random) 3x4 matrix of linear forms in $\PP^4$
  mentioned earlier.
Find the dimensions of the cohomology vector spaces of 1-forms $\Omega^1_X$.  Compute the Hodge diamond of $X$.
}
{\tiny\begin{verbatim}
  S = ZZ/32003[a,b,c,d,e]
  M = random(S^3, S^{4:-1})
  I = minors(3, M)
  dim I -- one more than the dimension as a projective variety
  codim I,  degree I
\end{verbatim}
  }
%%   hilbertPolynomial(S^1/I, Projective => false)
%%   for i from 0 to 10 list hilbertFunction(i, S^1/I)
%%   hilbertSeries I
%%   reduceHilbert oo

%%   X = variety I

%%   for i from 0 to 2 list HH^i OO_X
%%   R = S/I
%%   M1 = ker(vars S ** R)
%%   M2 = image((jacobian I) ** R)
%%   M = prune(M1/M2);
%%   Mp = coker lift(presentation M, S);
%%   ann Mp == I
%%   hilbertPolynomial(Mp, Projective => false) -- euler characteristic is -11, this matches!
%%   HH^0 sheaf Mp
%%   HH^1 sheaf Mp
%%   HH^2 sheaf Mp

%%   cotangentSheaf X -- wow, really long!!
%%   elapsedTime cotangentSheaf(X, MinimalGenerators => false)
%% \end{verbatim}}


\end{document}
